\documentclass[11pt]{article}

\usepackage[utf8]{inputenc}
\usepackage[ngerman]{babel}
\usepackage{csquotes}

\usepackage{fullpage}
\usepackage{setspace}
\usepackage{parskip}
\usepackage{titlesec}
\usepackage{mathptmx}
\usepackage{comment}

\usepackage{graphicx}
\usepackage[section]{placeins}
\usepackage[justification=centering]{caption}
\usepackage{wrapfig}

\usepackage{authblk}
\usepackage{hyperref}


\PassOptionsToPackage{
style=apa,
doi=false,
isbn=false,
eprint=false
}{biblatex}
\usepackage[backend=biber]{biblatex}
\addbibresource{ws1819_skribbl.bib}

\PassOptionsToPackage{hyphens}{url}

\makeatletter

\renewenvironment{abstract}
{{\bfseries\noindent{\abstractname}\par\nobreak}\footnotesize}
{\bigskip}

\renewenvironment{quote}
  {\begin{tabular}{|p{13cm}}}
  {\end{tabular}}

\titlespacing{\section}{0pt}{*3}{*1}
\titlespacing{\subsection}{0pt}{*2}{*0.5}
\titlespacing{\subsubsection}{0pt}{*1.5}{0pt}

\begin{document}

\begin{titlepage}
   \begin{center}
       \vspace*{1cm}

       \Huge
       Skribbl.AI Projektdokumentation (IQ vs. KI)
       \vspace{1.5cm}

       \includegraphics[width=0.4\textwidth]{images/logo_htw.jpg}

       \vspace{1.0cm}
       \includegraphics[width=0.2\textwidth]{images/logo_skribbl.png}
       \vspace{1.0cm}
       \LARGE

       Josefine Sophie Busch, Malin Dulkies, Rachel Escueta, Tim-Niklas Heise, Ninoslav Kjireski, Anastasia Litvina, Anh Quang Vu-Tuyen

       \vfill

       Projektarbeit \\
       Prof. Dr. Klaus Jung\\
       Wintersemester 2018/2019\\

       \vspace{0.8cm}

       Hochschule für Technik und Wirtschaft\\
       Berlin\\

   \end{center}
\end{titlepage}

\pagebreak
\tableofcontents
\pagebreak
\listoftables
\listoffigures
\pagebreak

\section{Einleitung}
\label{chap: Einleitung}
\subsection{Leitfaden}
\begin{wrapfigure}{R}{0.3\textwidth}
\centering
\includegraphics[width=0.25\textwidth]{images/logo_skribbl.png}
\caption{\label{fig:skribblLogo}Skribbl.AI Logo}
\end{wrapfigure}
Die Idee begann mit IQ gegen KI. Mit dem Einfall, was würde eigentlich passieren, wenn eine künstliche Intelligenz ein von einem Menschen gemaltes Bild  erkennen sollte? Neuronale Netzwerke erbringen beeindruckende Leistungen bei der Erkennung fotorealistischer Aufnahmen. Sie helfen sogar Pathologen bei der Erkennung von Krebszellen \parencite{ElizabethDougherty2018}. Was also würde  passieren, wenn wir eine künstliche Intelligenz mit den Kritzeleien einer Student*in konfrontieren?

\begin{comment}
 (Was ist das Projekt, wie ist es zustande gekommen, was ist das Ziel)
 
 TODO Liste für aufzählung in 1.1
\end{comment}
 
Das Semesterprojekt ist ein Pflichtmodul des Internationale Medieninformatik Studienganges der HTW Berlin, welches im 5. Semester angesetzt ist. \\
In einer Gruppe von 5-8 zufällig gewählten Studierenden werden Konzepte entwickelt und dann in Anwendungen umgesetzt, dabei können die Themen, an denen die Studierenden arbeiten, aus Bereichen der Wirtschaft oder aus einem der verschiedenen Forschungsbereiche des Studienganges kommen. \\
Die Studierenden werden dabei von einer Lehrkraft bzw. einer verantwortlichen Person aus dem wirtschaftlichen Unternehmen betreut.\\
Das Projekt ist in drei Bereiche unterteilt: 
\begin{enumerate}
	\item der Themenvergabe/Konzeptausarbeitung 1.3.1
	\item dem Projektmanagement 1.3.2
	\item der Durchführung 1.3.3
\end{enumerate}
dazu jedoch im nachfolgenden Kapitel mehr.\\
Die IMI Showtime Messe, auf welcher die Gruppen ihre Ergebnisse präsentieren, bildet den Abschluss des Semesterprojektes, dabei wird 70 Prozent (?) der Note anhand der Präsentation, dem Auftreten als Gruppe und dem Stand des Fortschritts festgesetzt.

\subsection{Aufbau des Projektes}
\label{chap: Aufbau}

Das Semesterprojekt ist ein Pflichtmodul des Internationale Medieninformatik Studienganges der HTW Berlin, welches im 5. Semester angesetzt ist.

In einer Gruppe von 5-8 zufällig gewählten Studierenden werden Konzepte entwickelt und dann in Anwendungen umgesetzt, dabei können die Themen, an denen die Studierenden arbeiten, aus Bereichen der Wirtschaft oder aus einem der verschiedenen Forschungsbereiche des Studienganges kommen.
Die Studierenden werden dabei von einer Lehrkraft bzw. einer verantwortlichen Person aus dem wirtschaftlichen Unternehmen betreut.

Das Projekt ist in drei Bereiche unterteilt: 1. der Themenvergabe/Konzeptausarbeitung, 2. dem Projektmanagement und 3. der Durchführung, dazu jedoch unter 1.2 Aufbau des Projektes mehr.

Die IMI Showtime Messe, auf welcher die Gruppen ihre Ergebnisse präsentieren, bildet den Abschluss des Semesterprojektes, dabei wird 70 Prozent (?) der Note Anhand von der Präsentation, dem Auftreten als Gruppe und dem Stand des Fortschrittes festgesetzt.
\subsection{Aufbau des Projektes}

Das Projekt ist, wie unter \autoref{chap: Einleitung} 1.1 Leitfaden schon erwähnt, in drei Bereiche unterteilt, welche im folgenden genauer und im Kontext unseres Projektes erläutert werden.

\subsubsection{ Themenvergabe/Konzeptausarbeitung }
\label{chap: Themenvergabe}

Wir haben das Projekt KI vs IQ erhalten, welches von Prof. Jung 		betreut wurde.\\
Die Vorgabe war, eine spielerische Anwendung zu entwickeln, 			in der ein Mensch gegen ein neuronales Netz antritt, welches 			dazu trainiert wurde, Bilder zu erkennen.\\
Mehr dazu im \autoref{chap: Grundlagen}
	
\subsubsection{  Projektmanagement }
\label{chap: Projektmanagement}
Die Belegung des Semesterprojektes erfolgt mit einer 						automatischen Belegung des Projektmanagement Moduls, 				welches sich auf das agile Arbeiten mit Scrum in Software 				Projekten beschäftigt. Dem entsprechend und der Tatsache 				zugrundeliegend, dass einige Gruppenmitglieder bereits 					Erfahrung mit Scrum hatten, haben wir uns dafür entschieden, im 	Projekt mit Scrum zu arbeiten bzw. einer unserer 						Bedürfnisse und Gegebenheiten angepassten Version von 				Scrum.\\
Als unterstützende Tools für das Projektmanagement des Projektes haben wir 	Trello genutzt und zum Gruppenmanagement bzw. als 		Kommunikations-Tool Slack.

\subsubsection{ Durchführung }
Die Durchführungsphase setzt sich dann konkret mit der 					Entwicklung der Anwendung auseinander.\\
Wir entschieden uns dafür eine Website zu bauen, auf welcher 			unser Spiel dann läuft.\\
Wir haben HTML, CSS und Javascript für das entwickeln der 			Webumgebung und des Spieles genutzt und Python für das 				Trainieren des neuronalen Netzwerkes.\\
Mehr dazu im \autoref{chap: Anforderungen}
1. Themenvergabe/Konzeptausarbeitung
	Wir haben das Projekt KI vs IQ erhalten, welches von Prof. Jung 		betreut wurde.
	Die Vorgabe war, eine spielerische Anwendung zu entwickeln, 			in der Mensch gegen neuronales Netz antritt, welches dazu 				trainiert wurde, Strichzeichnungen zu erkennen.
	Der Mensch sollte etwas zeichnen und die AI sollte dieses 				erkennen und das ganze als Spiel.
	Wir entschieden uns dafür eine Website zu bauen, auf welcher 			unser Spiel dann läuft. \\
	Mehr dazu im \autoref{chap: Grundlagen}

2. Projektmanagement
	Die Belegung des Semesterprojektes erfolgt mit einer 						automatischen Belegung des Projektmanagement Moduls, 				das sich auf das agile Arbeiten mit Scrum in Software 				Projekten beschäftigt. Dem entsprechen und der Tatsache 				zugrundeliegend, dass einige Gruppenmitglieder bereits 					Erfahrung mit Scrum hatten, haben wir uns dafür entschieden, im 	Projekt unter(?) Scrum zu arbeiten bzw. einer unserer 						Bedürfnisse und Gegebenheiten angepassten Version von 				Scrum.
	Als unterstützenden Tools zum managen des Projektes haben wir 	Trello genutzt und zum managen der Gruppe bzw. als 						Kommunikations-Tool Slack.

3. Durchführung
	Die Durchführungsphase setzt sich dann konkret mit der 					Entwicklung der Anwendung zusammen.
	Wir haben HTML, CSS und Javascript für das entwickeln der 			Webumgebung und des Spieles genutzt und Python für das 				entwickeln des neuronalen Netzwerkes.
	Mehr dazu im \autoref{chap: Anforderungen}

\section{Grundlagen}
\label{chap: Grundlagen}
\subsection{Design}
\subsubsection{Findung Und SketchR}

Wissend, dass wir ein Spiel kreieren wollten, welches sowohl Bilderkennung durch ein neuronales Netzwerk garantiert als auch den Spielspaß für den Spieler, blieben wir bevor weitere Überlegungen angefangen werden konnten, direkt an Namen sowie Konzept des Projekts hängen.
Abgeleitet durch das englische Wort 'to scribble' mit ein paar wenigen Änderungen um den Namen für das deutsche Publikum besser aussprechbar zu machen, wurde 'Skribbl' geboren. Das Kürzel '.AI' hingegen war für uns als Team sehr offensichtlich. Nicht nur wäre es eine wundervolle URL, wenn jemand das Projekt veröffentlichen wollen würde, sondern sie zeigt ebenfalls die nahe Verbindung zum neuronalen Netzwerk.

Folgend kam die Entscheidung dahingehend, was für eine Art von Spiel wir erschaffen wollten. Eine Zeichnung sollte erkannt werden, aber die Möglichkeiten dies umzusetzen waren zu umfangreich und so mussten wir uns auf ein Konzept einigen. Eines, welches nicht zu \textit{viel} beinhaltete und den Rahmen nicht sprengen würde, allerdings auch nicht zu einfach ausfallen würde.

Letztendlich entschieden wir uns dafür erst einmal zu recherchieren, ob es nicht vielleicht bereits jemanden gab der ein ähnliches Konzept umgewandelt hat. Spiel oder nicht war dabei egal.
Dabei stießen wir auf zwei Seiten verschiedene Programme (SketchR und Quick, Draw! [Google]). Geholfen haben diese bei unserer Entscheidung in welche Richtung das Spiel sich entwickeln sollte definitiv. Nach einigen Diskussionen entschieden wir uns schließlich für ein Zeichen-Spiel mit integriertem Zeitfaktor, bei dem die Künstliche Intelligenz das Bild mit Hilfe eines Treppchen Systems (ist das gesuchte Wort wahrscheinlicher als alle anderen, somit ist es auf Platz 1), erkennen sollte.

\subsubsection{Mobile First}

Ein solches Projekt, mit Spiel-Charakter, wie man sie aus jedem App-Store kennt; kurze Level für ein Kurzweiliges Spielerlebnis machten eines schnell klar: Unsere Priorität lag auf Mobile-First.
Dabei musste der Touchscreen, die unterschiedlichen Bildschirmgrößen sowie Funktionen und Hürden der verschiedenen Browser und Betriebssysteme mit eingeplant werden. Dennoch war es für uns keine schwere Entscheidung den Desktop vorerst außen vor zu lassen und sicher zu stellen, dass unser Spiel wenigstens auf einem Smartphone spielbar ist.

\subsubsection{Accessibility}

Accessibility war ein Punkt, der zwar erst ein wenig später in unserem Projekt auftauchte, aber dennoch nicht von kleiner Bedeutung war. Mit dem sehr visuellen Thema des Zeichnens und der heutigen Einstellung zu Accessibility Themen, war uns schnell klar, dass wir unser Spiel so zugänglich wie möglich für alle Nutzer machen mussten.

Ein Styleguide (Styleguide anhängen?) wurde erstellt, in welchem jede Farbe und Schriftgröße so getestet wurde, dass sie den Werten der W3C Verordnung entspricht.
Gleichzeitig wurden noch andere Details beachtet, wie zum Beispiel welche Coding-Regeln einzuhalten waren.


\subsection{Frontend}
\subsection{Neuronales Netzwerk}
\subsubsection{Dataset}
\section{Anforderungen}
\label{chap: Anforderungen}
\subsection{Produdt Vision}
\label{sec:productVision}
Unsere Produkt Vision ist, bis zum Ende des Wintersemesters 2018/2019 ein kurzweiliges Mini-Malspiel mit einem Einzelspielermodus entwickelt zu haben. Es soll speziell für mobile Endnutzer aller Altersklassen konzipiert sein.
\subsubsection{Die Kurzweiligkeit}
Die Kurzweiligkeit unseres Spiels ist offensichtlich eine der Eigenschaften, die am schwierigsten einzuschätzen sind. Um dennoch dieses Ziel nicht aus den Augen zu verlieren, einigten wir uns darauf, alle Tickets, Erweiterungen und Ideen, die sich mit dem Spielkonzept beschäftigen würden, auf ihren Unterhaltungswert zu prüfen.
\subsubsection{Das Mini-Spiel}
Das Spiel sollte runden basiert sein und jede Runde sollte nur wenige Minuten dauern. Damit wollen wir gewährleisten, dass es gut als Zeitvertreib während beispielsweise dem Fahren öffentlicher Verkehrsmittel oder dem Warten in der Schlange im Coffeeshop gespielt werden kann. Konkret sollte eine Runde nicht länger als 2-3 Minuten dauern.
\subsubsection{Das Mal-Spiel}
Um das Spiel zu spielen, wird die Spieler*in als Hauptaufgabe Zeichnen. Dabei sollen übliche Werkzeuge wie ein \textit{undo} Knopf, zum Entfernen des zuletzt gemalten Strichs und ein \textit{clear} Knopf, zum Entfernen der gesamten Zeichnung, implementiert werden. Weiter wird die Spieler*in direkt mit dem Finger auf dem Bildschirm beziehungsweise mit Hilfe der Maus Zeichnen können. Mehr dazu in Kapitel~\ref{sec:spielkonzept}.
\subsubsection{Der Einzelspielermodus}
Skribbl.AI soll von einer Person einzeln gespielt werden können. Über Gewinn oder Verlust soll die vergangene Zeit entscheiden, welche die Spieler*in benötigt haben wird, um das Geforderte zu zeichnen. Mehr hierzu ebenfalls in Kapitel~ref{sec:spielkonzept}.
\subsubsection{Für mobile Geräte}
Um Skribbl.AI auf vielen verschiedenen mobilen Endgeräten spielen zu können ohne verschiedene Programme für unterschiedliche Betriebssysteme entwickeln zu müssen, entschieden wir uns für eine webbasierte Anwendung. Das Spiel sollte dann auf einem, über das Internet erreichbaren Host, veröffentlicht werden, sodass es umstandslos mit übliche Browsern gespielt werden kann.
\subsubsection{Für alle Altersklassen}
Das Spiel sollte für alle Alterklassen interessant und spielbar sein. Neben der Eigenschaft, dass eine Runde nur eine bestimmte Dauer haben sollte, würden hier hauptsächlich die Begriffe, die der Spieler malen sollte, ausschlaggebend sein. Siehe hierzu Kapitel~ref{sec:spielkonzept}.
\subsection{Spielkonzept}
\label{sec:spielkonzept}
\begin{figure}[ht]
	\centering
	\includegraphics[width=1\textwidth]{images/blindtesting.png}
	\caption{\label{fig:blindtestingtable}Ausschnitt der Ergebnisse der Zeittests}
\end{figure}
Um die, in Kapitel~ref{sec:productVision} genannten Eigenschaften in unserem Spiel umzusetzen entwarfen wir verschiedene Spielkonzepte. Das Konzept, auf welches wir uns letzendlich einigten sieht wie folgt aus.

Die Spieler*in sollte zu Beginn einer Spielrunde für wenige Sekunden ein Wort aus einem Pool von Wörtern gezeigt bekommen. Dieses Wort ist dann innerhalb eines festgelegten Zeitraums zu Zeichnen. Die optimale Zeit, die eine Runde dauert, sollte im späteren Verlauf des Projekts ermittelt werden. Hierzu haben alle Teammitglieder 24 verschiedene Begriffe gezeichnet. Mit Hilfe der Ergebnisse, ein Ausschnitt dieser ist zu sehen in Abbildung \ref{fig:blindtestingtable}, bestimmten wir eine angebrachte Zeit.
Jedesmal wenn die Spieler*in den "Stift absetzt", das bedeutet in unserem Fall, den Finger hebt um neu anzusetzen, soll das gezeichnete an das neuronale Netzwerk übergeben werden. Dieses berechnet dann die Ergebnisse. Die fünf Ergebnisse mit den höchsten Vorhersagewahrscheinlichkeiten sollen auf dem Spielbildschirm angezeigt werden. Für die Anzeige der fünf wahrscheinlichsten Wörter soll ein Balkendiagramm verwendet werden, welches das Wort und die zugehörige Wahrscheinlichkeit in Prozent zeigt. Außerdem soll, falls das korrekte Wort in der Auswahl enthalten ist, dieses farblich hervorgehoben werden. Am oberen Rand des Spielbildschirms soll ein farblich hervorgehobener Balken anzeigen, wie viel Zeit schon vergangen ist.
Wenn die Spieler*in es schafft, eine Zeichnung innerhalb der vorgegebenen Zeit zu erstellen, die das neuronale Netzwerk als das vorgegebene Wort erkennt, dann ist die Runde gewonnen. Das bedeutet, dass das gesuchte Wort in den Vorhersagen des neuronalen Netzwerks die höchste Wahrscheinlichkeit im Vergleich mit den anderen Wörtern haben muss.
Verloren ist das Spiel, wenn dies der Spieler*in nicht in der vorgegebenen Zeit gelingt.
In beiden Fällen hat die Spieler*in dann die Möglichkeit eine neue Runde mit einem neuen Wort zu starten.
\section{Umsetzung, Implementierung, Ausarbeitung}
\section{Spielbeschreibung, Ziel}
\section{Bewertung}
vgl. Anforderungen mit Ergebnis

\pagebreak
\printbibliography
\end{document}
