\documentclass[11pt]{article}

\usepackage[utf8]{inputenc}
\usepackage[ngerman]{babel}
\usepackage{csquotes}

\usepackage{fullpage}
\usepackage{setspace}
\usepackage{parskip}
\usepackage{titlesec}
\usepackage{mathptmx}

\usepackage{graphicx}
\usepackage[section]{placeins}

\PassOptionsToPackage{
style=apa,
doi=false,
isbn=false,
eprint=false
}{biblatex}
\usepackage[backend=biber]{biblatex}
\addbibresource{library.bib}

\PassOptionsToPackage{hyphens}{url}

\makeatletter

\renewenvironment{abstract}
{{\bfseries\noindent{\abstractname}\par\nobreak}\footnotesize}
{\bigskip}

\renewenvironment{quote}
  {\begin{tabular}{|p{13cm}}}
  {\end{tabular}}

\titlespacing{\section}{0pt}{*3}{*1}
\titlespacing{\subsection}{0pt}{*2}{*0.5}
\titlespacing{\subsubsection}{0pt}{*1.5}{0pt}

\usepackage{authblk}

\usepackage[colorlinks = false]{hyperref}

\begin{document}

\begin{titlepage}
   \begin{center}
       \vspace*{1cm}

       \Huge
       Skribbl Projektdokumentation (IQ vs. KI)
       \vspace{2.0cm}

       \includegraphics[width=0.4\textwidth]{logo.jpg}

       \vspace{1.5cm}
       \LARGE

       Josefine Sophie Busch, Malin Dulkies, Rachel Escueta, Tim-Niklas Heise, Ninoslav Kjireski, Anastasia Litvina, Anh Quang Vu-Tuyen

       \vfill

       Projektarbeit \\
       Prof. Dr. Klaus Jung\\
       Wintersemester 2018/2019\\

       \vspace{0.8cm}

       Hochschule für Technik und Wirtschaft\\
       Berlin\\

   \end{center}
\end{titlepage}

\pagebreak
\tableofcontents
\pagebreak
\listoftables
\listoffigures
\pagebreak

\section{Einleitung}
\subsection{Leitfaden}
 (Was ist das Projekt, wie ist es zustande gekommen, was ist das Ziel)
\subsection{Aufbau des Projektes}
    (Wir haben uns für agile, Scrum usw. entschieden)    1.2.1. Trello, Slack

\section{Grundlagen}
\subsection{Design}
2.1 Design( - Findung)(Hier kommt rein wie wir von AI vs KI zu Skribbl.AI gekommen sind)
Mobile First
Accessibility
Orientiert an SketchR
\subsection{Frontend}
\subsection{Neuronales Netzwerk}
\subsubsection{Dataset}
\section{Anforderungen}
Genauere Productvision, Detailierter
\section{Umsetzung, Implementierung, Ausarbeitung}
\section{Spielbeschreibung, Ziel}
\section{Bewertung}
vgl. Anforderungen mit Ergebnis

\printbibliography
\end{document}
